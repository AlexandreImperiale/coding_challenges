\documentclass[11pt,a4paper]{article}


%============================================================================================
% Language
%============================================================================================
\usepackage[utf8]{inputenc}

%============================================================================================
% Page setup
%============================================================================================
\usepackage{fullpage}
\usepackage[affil-it]{authblk}


%============================================================================================
% Symbols
%============================================================================================
\usepackage{amsmath}
\usepackage{amssymb}


%============================================================================================
% User defined cmds
%============================================================================================
\title{Solving the ``star system'' problem using Voronoi tessellation}
\author{}

\begin{document}

\bibliographystyle{apalike}
% \maketitle

\subsubsection*{Problem analysis}

\begin{itemize}
\item  We consider the star system as a collection of 2D points $\mathcal{P} = \{P_i\}_{i=0}^{n-1}$. Assuming that we can build an adjacency relation between points, a path between two points $P_i,~P_j \in \mathcal{P}$ is obtained as a succession of displacements within the adjacency graph. With this assumption, we seek $d$ the minimal distance allowing to browse every paths between two points in the point set. \\
\item The first task is then to propose a suitable adjacency relation between points and we propose to use in the following the Voronoi diagram associated to the set of points. The Voronoi diagram benefits from numerous fundamental properties --~see for instance \cite{aurenhammer1991voronoi}, \cite{o1998computational}~-- and connecting every adjacent points in this diagram leads to the so-called Delaunay triangulation. \\
\item From the Delaunay triangulation, a potential solution would be to define the sought minimal distance as the maximal edge length in the triangulation. While this would give a sufficient condition to browse the adjacency graph, it would not be optimal. To illustrate this, we can consider the most simple counter-example of three points $\mathcal{P}~=~\{(0, 0),~(1, 0), ~(0, 1)\}$, leading to one triangle with maximal edge length of $\sqrt{2}$, while the optimal solution clearly is $d=1$.\\
\item Therefore, we face the problem of finding a subgraph of the Delaunay edges graph which is necessarily: (1) spanning, since we need to be able to reach any point within the point set; (2) made of edges with minimal length. \\
\item In order to build this subgraph, we can resort to an incremental approach. We first sort every edges by length, and add them into the subgraph under construction. This simple algorithm is actually, in essence, the Kruskal's algorithm \cite{kruskal1956shortest} used to build the Euclidian Minimal Spanning Tree (EMST). \\
\item From the Kruskal's algorithm, we can finally obtain the solution by simply define $d$ the minimal distance allowing to browse every paths between two points in the point set as the length of the last edge added in the EMST.
\end{itemize}


\subsubsection*{Complexity considerations} 

The various algorithm components can be broadly summarized as follows:
\begin{itemize}
	\item Compute, from the set of points, the Delaunay triangulation. Even though numerous approaches can be considered, we propose to perform this step using a standard Divide-and-Conquer strategy --~devise for instance in \cite{guibas1985primitives}~-- which has a $O(n\mathrm{log}(n))$ time complexity.\\
	\item The previous step leads to the construction of $O(n)$ edges in the Delaunay triangulation that need to be sorted using their associated length, from which can expect a $O(n\mathrm{log}(n))$ time complexity.\\
	\item Finally, implementing the Kruskal's can be done such that a $O(n\mathrm{log}(n))$ time complexity is achieved.
\end{itemize}
Hence, the overall time complexity of the algorithm is expected to be in $O(n\mathrm{log}(n))$, which satisfies the problem requirements.

\bibliography{star_system_doc}

\end{document}


